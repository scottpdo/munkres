\documentclass{article}

\title{Munkres Ch. 2: Topological Spaces and Continuous Functions}
\author{Scott Donaldson}
\date{Jul. 2024}
\usepackage{amsmath, amsthm, amsfonts, amssymb, enumitem, tabu, tikz}

\begin{document}

\maketitle

\section*{§12. Topological Spaces\\§13. Basis for a Topology}

\subsection*{1. (7/30/24)}

Let $X$ be a topological space; let $A$ be a subset of $X$. Suppose that for each $x \in A$ there is an open set $U$ containing $x$ such that $U \subset A$. Show that $A$ is open in $X$.

\begin{proof}
    From the definition of an open set in a topological space, we know that an arbitrary union of open sets is again an open set. We will show that $A$ is a union of open sets in $X$, and is therefore open.

    For each $x \in A$, there exists an open set $U_x$ containing $x$ that is a subset of $A$. We claim that $A = \bigcup_{x \in A} U_x$.

    Let $a \in A$. It is given that there exists a $U_a \subset A$ with $a \in U_x$. Therefore $a \in \bigcap_{x \in A} U_x$.

    Conversely, let $a \in \bigcap_{x \in A} U_x$. Then $a$ lies in some $U_x$ such that $U_x \subset A$, and so $a \in A$. Thus $A = \bigcup_{x \in A} U_x$.
\end{proof}

\subsection*{2. (7/30/24)}

Consider the nine topologies on the set $X = \{ a, b, c \}$ indicated in Example 1 of §12. Compare them; that is, for each pair of topologies, determine whether they are comparable, and if so, which is the finer.

\begin{proof}[Solution.]
    Let the nine topologies on $X$ be:

    \begin{center}
        \begin{tabular}{ c c }
            i) & $\{ \varnothing, X \}$ \\
            ii) & $\{ \varnothing, \{ a \}, \{ a, b \}, X \}$ \\
            iii) & $\{ \varnothing, \{ b \}, \{ a, b \}, \{ b, c \}, X \}$ \\
            iv) & $\{ \varnothing, \{ b \}, X \}$ \\
            v) & $\{ \varnothing, \{ a \}, \{ b, c \}, X \}$ \\
            vi) & $\{ \varnothing, \{ b \}, \{ c \}, \{ a, b \}, \{ b, c \}, X \}$ \\
            vii) & $\{ \varnothing, \{ a, b \}, X \}$ \\
            vii) & $\{ \varnothing, \{ a \}, \{ b \}, \{ a, b \}, X \}$ \\
            ix) & $\{ \varnothing, \{ a \}, \{ b \}, \{ c \}, \{ a, b \}, \{ b, c \}, \{ a, c \}, X \}$ \\
        \end{tabular}
    \end{center}
    Then:
    \begin{enumerate}[label=\roman{enumi})]
        \item is coarser than every other topology; ix) is finer than every other topology;
        \item is finer than vii), is not comparable to iii), iv), v), and vi), and is coarser than viii);
        \item is finer than iv) and vii), and is not comparable to v), vi), and viii);
        \item is not comparable to v) and vii), and is coarser than vi) and viii);
        \item is not comparable to vi), vii), or vii);
        \item is finer than vii) and is not comparable to viii); and
        \item is coarser than viii).
    \end{enumerate}
\end{proof}

\subsection*{3. (7/31/24)}

Show that the collection $\mathcal{T}_C$ given in Example 4 of §12 is a topology on the set $X$. Is the collection
\begin{equation*}
    \mathcal{T}_\infty = \{ U \mid X - U \text{ is infinite or empty or all of }X \}
\end{equation*}
a topology on $X$?

\begin{proof}
    Let $\mathcal{T}_C = \{ U \mid X - U \text{ is countable or is all of }X \}$. Now $\mathcal{T}_C$ certainly contains $\varnothing$ and $X$, since $X - \varnothing = X$ and $X - X = \varnothing$, a countable set.

    We must next show that arbitrary unions and finite intersections of elements of $\mathcal{T}_C$ are again elements of $\mathcal{T}_C$.

    Let $\{ U_\alpha \}$ be an indexed family of nonempty elements of $\mathcal{T}_C$. To show that $\bigcup U_\alpha$ is in $\mathcal{T}_C$, we compute:
    \begin{equation*}
        X - \bigcup U_\alpha = \bigcap (X - U_\alpha),
    \end{equation*}
    and since each $X - U_\alpha$ is countable, their intersection is also countable. Therefore an arbitrary union of elements of $\mathcal{T}_C$ lies in $\mathcal{T}_C$.

    Finally, let $U_1, U_2, ..., U_n$ be nonempty elements of $\mathcal{T}_C$ and consider $\bigcup_{i = 1}^n U_i$. We have:
    \begin{equation*}
        X - \bigcap_{i = 1}^n U_i = \bigcup_{i = 1}^n (X - U_i),
    \end{equation*}
    which is a finite union of countable sets, and is therefore countable. Thus $\bigcup_{i = 1}^n U_i$ is in $\mathcal{T}_C$, and so $\mathcal{T}_C$ is a topology on $X$.

    Next we consider $\mathcal{T}_\infty = \{ U \mid X - U \text{ is infinite or empty or all of }X \}$, and claim that it is not a topology on the set $X$. As a counterexample, suppose that
    \begin{align*}
        X &= \mathbb{R}, \\
        U_a &= \{ x \in X \mid x < 0 \}, \text{ and } \\
        U_b &= \{ x \in X \mid x > 0 \}.
    \end{align*}
    Then $U_a \cup U_b = \{ x \in X \mid x < 0 \text{ or } x > 0 \}$, that is, the nonzero real numbers, whose complement is $\{ 0 \}$, a finite set, and therefore the union of $U_a$ and $U_b$ is not a member of $\mathcal{T}_\infty$. It is therefore not a topology on $X$.
\end{proof}

\end{document}